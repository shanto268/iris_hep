\documentclass[10pt]{article}
\usepackage{amsmath}
\usepackage{graphicx}
\usepackage{float}
\usepackage{setspace}
\usepackage{times}
\usepackage{lipsum}
\usepackage[margin=0.9in]{geometry}
\usepackage{fancyhdr}
\pagestyle{fancy}
\usepackage{hyperref}
\usepackage{listings}
\usepackage{adjustbox}
\renewcommand*\contentsname{Table of Contents}
%\setlength\bibitemsep{1.5\itemsep}
\setlength{\parindent}{0em}

%\pagenumbering{roman}
%\setlength{\parskip}{1em}

\begin{document}
\begin{center}
    \Large{\textbf{Computationally Efficient High Resolution Muon Tomography through use of RNN, LSTM and Image Segmentation Techniques}} \\ 
    \vspace{1.5mm}
    \large{Sadman Ahmed Shanto*, \textit{Texas Tech University}} \\
    \large{PI: Nural Akchurin, \textit{Texas Tech University}}
\end{center}

\lhead{IRIS-HEP Undergraduate Fellowship Proposal}
\rhead{\textit{Nov 2020}}
\lfoot{*sadman-ahmed.shanto@ttu.edu}

\flushleft
\large{
\textbf{Duration:} Jan 2021 - May 2021 \\
\textbf{Category:} Analysis Systems (AS) \\
\textbf{Funding Period:} 13 weeks in Spring 2021}

\section{Project Background}

%Explain MT, talk about our work and paper in context of motivating the problem. Talk about current efforts and problems from lab and HEP community in general (ref the problem paper). 
%Problems: Parallelization not currently exploited (software fixes may be?),  ROI determination, thread balancing due to non-uniformity, inadequacy of Kalman Filter when it comes to sequential data, Quadratic Scaling, and data growth $>>$ computing power

Muon Tomography (MT) is a technique that utilizes muon scattering and muon absorption to generate images of large target objects such as buildings, volcanoes, and ancient archaeological structures \cite{paper, paper1, paper2, paper3, paper4, paper5}. Data collection takes a long time due to limited muon rates and large amounts of high quality data is needed for sharp image reconstruction. Incomplete or inefficient event information is often disregarded in traditional analysis. Such premature data processing means that potentially useful information is not utilized. In addition, the application of modern image reconstruction techniques has been slow in MT where the image quality can be significantly enhanced compared to ones produced by traditional tracking algorithms.

%Slow rates of successful muon events needed for efficient tomography makes the operation time for Muon Telescopes tediously long and generates copious amounts of data in the process. Most of this data is not of interest to the operator and is discarded. Such premature data processing means that important information about muon spectrum are not being utlized, thereby resulting in substandard tomograms. Furthermore, these resultant tomograms are usually limited to the resolution determined by the hardware configuration of the telescope \cite{problem}; The lack of advancments in software techniques to resolve such problems has made this issue worse. This motivates the need for some intelligent data processing regimes that make use of the great volume of data generated without compromising operating time and system resources. Moreover, application of Machine Learning techniques such as RNN, LSTM, and IS should be utilized to mitigate the influence of hardware capabilities that traditionally act as a limiting factor on the resolution. 

\vspace{2mm}

\hspace{10mm} Our group at Texas Tech University has been working on MT for the past few years (See a \href{https://www.symmetrymagazine.org/article/archaeology-meets-particle-physics}{\textit{Symmetry}} Article on our work). Last year, we developed a portable prototype telescope that was capable of generating images at a resolution of 50 milliradians \cite{mt_paper}. Presently, we are working on a next generation telescope to achieve $100$ times finer resolution by recycling some of the hardware from the previous prototype but with the aid of better reconstruction software. The sequential emission file generated by the telescope makes it an ideal candidate for training Recurrent Neural Networks (RNNs) which would recast the problem of predicting "next hits" as a regression problem\cite{rnn_paper, old}. Such a schema can be made further robust by using Long Short Term Memomy (LSTMs) networks to contextualize the entire data frame, thus, providing an additional constraint on the training regime for the RNNs. Image Segmentation (IS) may finally be used to generate pixel map and extract shape information of target object to add the final constraint layer for RNN training regime.

\section{Project Proposal}

%Explain RNN, LSTM, IS and use in MT. Refer to the papers. Link project to a funding category. Include Github links. Open Source and stand alone python package with use of TensorFlow, Modin. Project completed in LBK.

\vspace{2mm}
The primary goal is to publish a computationally efficient python package that implements the aforementioned ML techniques to generate robust and high resolution tomograms from muon telescope datasets. The software architecture will be designed prioritizing high parallelization, whenever admissible, to reduce run time analysis. I also plan to implement a thread balancing scheme through the use of LSTM networks to determine regions of interest a priori as a part of the calibration process of the telescope which would help diminish the effects of non-uniformity of stochastic scattered muon hits. Following this initial objective, I will develop an additional python package that would extend the functionality from the previous package to cover the functional requirements of general tomography. Throughout this project I will enhance my knowledge of ML, MT, and parallel computing. I will analyze the various techniques recommended in literature and report on their efficacy to our original goal. This project is carried out at the Advanded Detector Particle Laboratory at Texas Tech University under the supervision of Dr. Akchurin.

\section{Deliverables}

\begin{itemize}
    \item Creation of a comprehensive dedicated Muon Imaging Framework leveraging popular tools in ML.
    \item Development of a well documented python package with easy to use API with implementations of RNN, LSTM and IS for applications in general tomography.
    \item Final report and a publication on a study of parallelized ML schemes for faster and more efficient training of neural networks.
\end{itemize}

\section{Timeline}

I will commit most of my time to this project in Spring 2021, my last semester at TTU before gradutation.

\begin{itemize}
    \item \textbf{Week 1:} Project setup; Implement \textit{input data agnostic} framework for training RNNs and LSTMs (using TensorFlow) seperately. \\
    \item \textbf{Week 2-3:} Combine the various training frameworks into a parallelized/distributed set up; Design testbed for verifying effictiveness of multithreading and cohesiveness of various ML components. \\
    \item \textbf{Week 4-5:} Train the comprehensive regime on simulated ``ideal" muon data for calibration and conduct thorough unit validation tests; generate tomograms and compare with base case.\\
    \item \textbf{Week 6:} Train the program on prerecorded experimental muon data; start implementation of IS to extract pixel map and shape information; merge the IS functionality into the regular operation of the telescope.\\
    \item \textbf{Week 7-8:} Study the performance of IS in the tomogram generation process; implement the thread balancing scheme; calibrate and train all the different parts of the software together. \\
    \item \textbf{Week 9-10:} Start testing performance of comprehensive system in on field real time operation using the prototype detector; create wrappers for core functionality; finish code documentation and tutorials. \\
    \item \textbf{Week 11:} Create new package for general tomography.\\
    \item \textbf{Week 12:} Submit both packages to Python Package Index (PyPI); finalize and polish the manuscript.
    \item \textbf{Week 13:} Prepare final report with findings and publish report as appropriate. \\
\end{itemize}

%\begin{table}[H]
    %\centering
%\begin{adjustbox}{width=1\textwidth}
    %\label{tab:label}
    %\begin{tabular}{|c|c|}
    %\hline
    %\textbf{Weeks} & \textbf{Tasks} \\ [0.5ex]
    %\hline
    %1  & Project setup; Implementation of \textit{input data agnostic} framework for training RNNs and LSTMs (using TensorFlow) done seperately. \\
    %\hline
    %2 \& 3 &  asd  \hline
    %4 \& 5 & 
    %\hline
    %6 & 
    %\hline
    %7 \& 8 & 
    %\hline
    %9 \& 10 &     \hline
    %11 &     \hline
    %12 &  \\
    %\hline
    %13 &     \hline
    %\end{tabular}

%\end{adjustbox}
    %\caption{Timeline and plan for the project}
%\end{table}

\newpage
\section{Student Background}

I am a senior Applied Physics and Mathematics major with a Computer Science minor from Texas Tech University. I have been working at the Advanced Particle Detector Laboratory at Texas Tech University for the past two years on various Muon Tomography projects - \textit{developing Monte Carlo simulations using Geant4 to test experimental data integrity, designing custom Printed Circuit Boards (PCBs) for more efficient read out electronics, assisting with mechanical and electronic assembly of two different prototype Muon Telescopes, conducting statistical and image analysis on measured datasets}. Presently, I am studying and implementing ML techniques for MT. In addition to my work with the High Energy Physics group, I have also worked on developing simulation and analysis software for Autonomous Vehicle systems, parallelized calibration framework for traffic flow simulation software (see CV for more details) and contributed to various Open Source Quantum Computing projects. The proposed project will allow me to apply my knowledge of software design to solve a challenging problem and would serve as an excellent segue as I start my PhD journey in the Fall of 2021.

%\newpage

\begin{thebibliography}{1}
\bibitem{paper}
Bonomi, Germano.
\textit{Progress in Muon Tomography}.
The European Physical Society Conference on High Energy Physics. 10.22323/1.314.0609 (2017)

\bibitem{paper1}
L. W. Alvarez et al.
\textit{Science 167, 832–839 (1970)}

\bibitem{paper2}
J. Marteau et al.
\textit{Nucl. Instrum. Methods A 695 23-28 (2012)}

\bibitem{paper3}
K. Morishima et al.
\textit{Nature http://dx.doi.org/10.1038/nature24647 (2017)}

\bibitem{paper4}
S. Procureur et al.
\textit{Nucl. Instrum. Methods Phys. Res., Sect. A 878, 169–179 (2018)}

\bibitem{paper5}
H. Gómez et al.
\textit{Nucl. Instrum. Methods Phys. Res., Sect. A 936, 14–17 (2019)}

\bibitem{mt_paper} 
R. Perez, S. A. Shanto, M. Moosajee, and S. Cano.
\textit{High-Resolution Muography Using a Prototype Portable Muon Telescope}. 
Journal of Undergraduate Reports in Physics 30, 100006 (2020)

\bibitem{rnn_paper} 
Steven Farrell et al.
\textit{Novel deep learning methods for track reconstruction}.
4th International Workshop Connecting The Dots 2018 (CTD2018)

\bibitem{old} 
Peterson, C.
\textit{Track finding with neural networks}.
Nuclear Instruments and Methods in Physics Research Section A: Accelerators, Spectrometers, Detectors and Associated Equipment 279 (3) 537-545 (1989). 

\end{thebibliography}


\end{document}

